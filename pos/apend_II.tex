\chapter{Questionário}
\label{apend:2}

Caro empresário/gestor, estou realizando um estudo que pretende identificar fatores habilitadores para definição de um modelo de negócio eletrônico para empresas do arranjo produtivo local. O questionário visa identificar o quanto sua empresa está aderente a esses fatores. Para o êxito desta pesquisa é essencial que o questionário seja respondido na íntegra, até o dia 07 de novembro de 2019. Será mantida a confidencialidade com respeito a todas as informações fornecidas.

O número de empresas que divulga/comercializa produtos/serviços online é cada vez maior, mostrando que é uma oportunidade de negócio dinâmica e interessante, mas isso não quer dizer que seja simples manter um negócio online. Pesquisas mostram que o tempo de vida útil das empresas que comercializam seus produtos online é bem menor se comparado às empresas físicas.

Há inúmeros fatores considerados habilitadores para definição de um modelo de negócio eletrônico, a começar pelo posicionamento estratégico. Em geral, os fatores candidatos a potencializar o negócio no ambiente eletrônico estão presentes em áreas específicas da empresa, em práticas e processos, ou em funções de negócio, considerados como chaves para o sucesso de uma organização.

Esta pesquisa faz parte do Trabalho Final de Curso do Bacharelado em Sistemas de Informação (Instituto de Informática da Universidade Federal de Goiás). Todas as informações coletadas servirão única e exclusivamente para subsidiar as análises do trabalho acadêmico. Portanto, será mantida a confidencialidade e a divulgação ficará restrita aos grandes números (agregados), ou seja, sem a necessidade de vincular à sua empresa.

Ao final da pesquisa, o resultado será disponibilizado a todos que responderem este questionário.

\begin{table}[]
\begin{tabular}{l}
Orlando da Cruz Pereira Júnior \\
Graduando em Sistemas de Informação - INF/UFG \\
Orientador: Prof. Dr. Eliomar Araújo de Lima \\
Telefone: (62) 98478-1195 \\
E-mail: orlando\_junior@discente.ufg.br
\end{tabular}
\end{table}

%========================================================================
% Perguntas do questionário

\begin{table}[]
\begin{tabular}{l}
\textbf{SEÇÃO 1} \\
\\
1- Nome (Opcional) \\
\_\_\_\_\_\_\_\_\_\_\_\_\_\_\_\_\_\_\_\_\_\_\_\_\_\_\_\_\_\_\_\_\_\_\_\_\_\_\_\_\_\_\_\_\_\_\_\_\_\_ \\
\\
2- E-mail \\
\_\_\_\_\_\_\_\_\_\_\_\_\_\_\_\_\_\_\_\_\_\_\_\_\_\_\_\_\_\_\_\_\_\_\_\_\_\_\_\_\_\_\_\_\_\_\_\_\_\_ \\
\\
3- Empresa (Opcional) \\
\_\_\_\_\_\_\_\_\_\_\_\_\_\_\_\_\_\_\_\_\_\_\_\_\_\_\_\_\_\_\_\_\_\_\_\_\_\_\_\_\_\_\_\_\_\_\_\_\_\_ \\
\\
4- Cargo que ocupa na empresa \\
\_\_\_\_\_\_\_\_\_\_\_\_\_\_\_\_\_\_\_\_\_\_\_\_\_\_\_\_\_\_\_\_\_\_\_\_\_\_\_\_\_\_\_\_\_\_\_\_\_\_ \\
\\
5- Celular ou telefone de contato (Opcional) \\
\_\_\_\_\_\_\_\_\_\_\_\_\_\_\_\_\_\_\_\_\_\_\_\_\_\_\_\_\_\_\_\_\_\_\_\_\_\_\_\_\_\_\_\_\_\_\_\_\_\_ \\
\\
\begin{tabular}[c]{@{}l@{}}6- A empresa comercializa seus produtos/serviços em canais\\ eletrônicos (e-commerce)?\end{tabular} \\
(   ) Sim - Prossiga na seção 2, da pergunta 7 a 19. \\
(   ) Não - Prossiga na seção 3, da pergunta 7 a 16.
\end{tabular}
\end{table}

%=========================================================================
% Seção 2

\begin{table}[]
\begin{tabular}{l}
\textbf{SEÇÃO 2 - Comercializa produtos/serviços em canais eletrônicos (e-commerce)} \\
 \\
7- Tipicação da Empresa/Organização? \\
(   ) PME \\
(   ) Grande porte \\
(   ) Startup \\
(   ) Fintech \\
(   ) Encubadora/Aceleradora \\
 \\
8- Tempo de mercado \\
(   ) Menos de 1 ano \\
(   ) Menos de 3 anos \\
(   ) Menos de 6 anos \\
(   ) Menos de 15 anos \\
(   ) Mais de 15 anos \\
 \\
9- Área de Atuação da Empresa \\
(   ) Desenvolvimento de soluções de ERP, SIG, BPM \\
(   ) Suporte Técnico e Infraestrutura de TI \\
(   ) Assessoria e consultoria em TI \\
(   ) Outsourcing (BPO, KPO, ...) \\
(   ) Desenvolvimento de soluções Fullstack \\
(   ) Desenvolvimento de soluções mobile \\
(   ) Desenvolvimento de soluções de CRM, BI \\
\begin{tabular}[c]{@{}l@{}}(   ) Desenvolvimento de soluções de Cloud, Big Data, Data Science, Machine Learning e\\ Deep Learning\end{tabular} \\
(   ) Certicadora/Homologadora \\
(   ) Capacitação e Qualicação prossional \\
(   ) Mentoring e/ou Coaching \\
(   ) Comércio de suprimentos de informática \\
(   ) Outro: \_\_\_\_\_\_\_\_\_\_\_\_\_\_\_\_\_\_\_\_\_\_\_\_\_\_ \\
 \\
\begin{tabular}[c]{@{}l@{}}10- Como você avalia seu nível de conhecimento em tecnologia da\\ informação?\end{tabular} \\
(   ) Mínimo \\
(   ) Razoável \\
(   ) Bom \\
(   ) Superior \\
 \\
\begin{tabular}[c]{@{}l@{}}11- A empresa terceiriza alguma função especíca, recurso ou\\ infraestrutura de TI?\end{tabular} \\
(   ) Sim \\
(   ) Não \\
\end{tabular}
\end{table}

%===============================

\begin{table}[]
\begin{tabular}{l}
\begin{tabular}[c]{@{}l@{}}12- Em quais plataformas a empresa divulga/comercializa\\ produtos/serviços on-line?\end{tabular} \\
(   ) Plataforma própria (Desenvolvida e mantida pela empresa) \\
(   ) Plataforma Terceirizada (Loja integrada, WIX, Tray, outros) \\
(   ) Redes sociais (Facebook, Instagram, outros) \\
(   ) Marketplace (Google Play Store, Mercado Livre, B2W, outros) \\
(   ) Outros. \\
 \\
\begin{tabular}[c]{@{}l@{}}13- Utiliza a modalidade Cross-docking na empresa, em que, por meio de\\ um acordo com fornecedores, a loja vende antes de comprar e\\ estabelece o prazo de entrega, considerando o tempo de reposição do\\ fornecedor?\end{tabular} \\
(   ) Sim \\
(   ) Não \\
 \\
\begin{tabular}[c]{@{}l@{}}14- A empresa mapeia o perfil de seu consumidor e atua em um\\ mercado de nicho, segmentado?\end{tabular} \\
(   ) Sim \\
(   ) Não \\
 \\
\begin{tabular}[c]{@{}l@{}}15- Realiza a análise ambiental e promove estudos de redesenho\\ (pivotar) dos processos da empresa regularmente?\end{tabular} \\
(   ) Sim \\
(   ) Não \\
 \\
\begin{tabular}[c]{@{}l@{}}16- A empresa utiliza algum instrumento (indicador) que monitora e\\ acompanha o nível de desempenho (KPIs, satisfatório ou insatisfatório)\\ e o nível de atingimento dos objetivos organizacionais (KRIs)?\end{tabular} \\
(   ) Sim \\
(   ) Não \\
 \\
\begin{tabular}[c]{@{}l@{}}17- A empresa faz parte de alguma comunidade empresarial, seja ela\\ cooperativa ou associativa?\end{tabular} \\
(   ) Sim \\
(   ) Não \\
 \\
\begin{tabular}[c]{@{}l@{}}18- A empresa atua na criação de conteúdo relevante para o seu público\\ alvo e o distribui por meio de blogs, redes sociais e campanhas de e-mail\\ marketing?\end{tabular} \\
(   ) Sim \\
(   ) Não \\
 \\
\begin{tabular}[c]{@{}l@{}}19- A empresa faz o acompanhamento durante o período de pós-venda,\\ monitorando o feedback dos clientes na internet, tanto em redes sociais\\ como em sites de reclamação, para poder melhorar um produto/serviço?\end{tabular} \\
(   ) Sim \\
(   ) Não
\end{tabular}
\end{table}

%=========================================================================
% Seção 3

\begin{table}[]
\begin{tabular}{l}
\textbf{SEÇÃO 3 - Não comercializa produtos/serviços on-line} \\
 \\
7- Tipicação da Empresa/Organização? \\
(   ) PME \\
(   ) Grande porte \\
(   ) Startup \\
(   ) Fintech \\
(   ) Encubadora/Aceleradora \\
 \\
8- Tempo de mercado \\
(   ) Menos de 1 ano \\
(   ) Menos de 3 anos \\
(   ) Menos de 6 anos \\
(   ) Menos de 15 anos \\
(   ) Mais de 15 anos \\
 \\
9- Área de Atuação da Empresa \\
(   ) Desenvolvimento de soluções de ERP, SIG, BPM \\
(   ) Suporte Técnico e Infraestrutura de TI \\
(   ) Assessoria e consultoria em TI \\
(   ) Outsourcing (BPO, KPO, ...) \\
(   ) Desenvolvimento de soluções Fullstack \\
(   ) Desenvolvimento de soluções mobile \\
(   ) Desenvolvimento de soluções de CRM, BI \\
\begin{tabular}[c]{@{}l@{}}(   ) Desenvolvimento de soluções de Cloud, Big Data, Data Science, Machine Learning e\\ Deep Learning\end{tabular} \\
(   ) Certicadora/Homologadora \\
(   ) Capacitação e Qualicação prossional \\
(   ) Mentoring e/ou Coaching \\
(   ) Comércio de suprimentos de informática \\
(   ) Outro: \_\_\_\_\_\_\_\_\_\_\_\_\_\_\_\_\_\_\_\_\_\_\_\_\_\_ \\
 \\
\begin{tabular}[c]{@{}l@{}}10- Como você avalia seu nível de conhecimento em tecnologia da\\ informação?\end{tabular} \\
(   ) Mínimo \\
(   ) Razoável \\
(   ) Bom \\
(   ) Superior \\
 \\
\begin{tabular}[c]{@{}l@{}}11- A empresa terceiriza alguma função especíca, recurso ou\\ infraestrutura de TI?\end{tabular} \\
(   ) Sim \\
(   ) Não \\
 \\
\end{tabular}
\end{table}

%===============================

\begin{table}[]
\begin{tabular}{l}
\begin{tabular}[c]{@{}l@{}}12- A empresa trabalha de forma que não precisa armazenar produtos
em estoque por \\muito tempo (just-in-time)? \end{tabular} \\
(   ) Sim \\
(   ) Não \\
 \\
\begin{tabular}[c]{@{}l@{}}13- A empresa mapeia o perfil de seu consumidor e atua em um\\ mercado de nicho, segmentado?\end{tabular} \\
(   ) Sim \\
(   ) Não \\
 \\
\begin{tabular}[c]{@{}l@{}}14- Realiza a análise ambiental e promove estudos de redesenho\\ (pivotar) dos processos da empresa regularmente?\end{tabular} \\
(   ) Sim \\
(   ) Não \\
 \\
\begin{tabular}[c]{@{}l@{}}15- A empresa utiliza algum instrumento (indicador) que monitora e\\ acompanha o nível de desempenho (KPIs, satisfatório ou insatisfatório)\\ e o nível de atingimento dos objetivos organizacionais (KRIs)?\end{tabular} \\
(   ) Sim \\
(   ) Não \\
 \\
\begin{tabular}[c]{@{}l@{}}16- A empresa faz parte de alguma comunidade empresarial, seja ela\\ cooperativa ou associativa?\end{tabular} \\
(   ) Sim \\
(   ) Não \\
\end{tabular}
\end{table}