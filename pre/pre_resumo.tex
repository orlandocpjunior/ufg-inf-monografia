\chaves{e-business, empreendedorismo digital, posicionamento estratégico on-line para PME´s.}

\begin{resumo} 
Este trabalho tem como objetivo identificar os fatores habilitadores para definição de um modelo de negócio eletrônico para empresas de pequeno e médio porte. Pesquisas mostram que a grande maioria dos empreendimentos virtuais no Brasil são pequenas e médias empresas e que a maioria delas encerram suas atividades em um curto período de tempo. A importância da pesquisa torna-se evidente diante do aumento do número de empresas que estão ingressando neste mercado e do pouco conteúdo bibliográfico existente. A revisão bibliográfica identificou alguns fatores habilitadores, dentre eles, o conhecimento em tecnologia da informação; a terceirização; o uso dos marketplaces; gerenciamento de estoque com o método cross docking; atuação em um mercado de nicho; utilização de instrumento indicador (KPIs, KRIs), e por fim, a realização de análise ambiental, promovendo estudos de redesenho dos processos. Foi realizada uma pesquisa com o público alvo que revelou que as empresas com maior tempo de mercado possuem gestores com nível de conhecimento em tecnologia da informação “Bom” ou “Superior”, utilizam algum instrumento indicador (KPIs, KRIs) e fazem o acompanhamento durante o período de pós-venda, monitorando o feedback dos clientes na internet. Este trabalho contribui ao meio acadêmico pela abordagem utilizada, uma vez que a pesquisa buscou obter a visão dos proprietários das organizações, e com isso, permitindo o avanço do conhecimento na área.
\end{resumo}
