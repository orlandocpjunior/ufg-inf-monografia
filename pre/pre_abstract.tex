\keys{e-business, digital entrepreneurship, online strategic positioning for SMEs.}

\begin{abstract}{Enabling factors for the definition of an electronic business model.}
This paper aims to identify the enabling factors for the definition of an electronic business model for small and medium sized companies. Research shows that the vast majority of virtual enterprises in Brazil are small and medium enterprises and most of them close down in a short period of time. The importance of research becomes evident given the increasing number of companies that are entering this market and the little existing bibliographic content. The literature review identified some enabling factors, including knowledge of information technology; outsourcing; the use of marketplaces; inventory management with the cross docking method; acting in a niche market; use of indicator instrument (KPIs, KRIs), and finally, the accomplishment of environmental analysis, promoting studies of process redesign. A survey was conducted with the target audience that revealed that companies with the longest market have managers with “Good” or “Higher” information technology knowledge, use some indicator tool (KPIs, KRIs) and follow up during the after-sales period by monitoring customer feedback on the internet. This work contributes to the academic environment by the approach used, since the research sought to obtain the vision of the owners of organizations, and thus, allowing the advance of knowledge in the area.
\end{abstract}
