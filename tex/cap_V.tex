\chapter{Conclusão}
\label{cap:conclusion}

Este trabalho teve como objetivo principal identificar quais são os fatores habilitadores candidatos para definição de um modelo de negócio eletrônico para empreendimentos de pequeno e médio porte que atuam no segmento de tecnologia da informação. Inicialmente efetuou-se uma revisão da literatura sobre os empreendimentos de pequeno e médio porte - PME, a definição sobre os negócios digitais, o modelo de negócio segundo a abordagem de Alexander Osterwalder,  partindo-se posteriormente para a etapa de coleta de dados fundamentada primordialmente na pesquisa com uso de questionários. 

A revisão da literatura mostrou que não há uma definição amplamente aceita sobre as PMEs no Brasil e a literatura continua com dificuldades em defini-las. Ainda assim, o princípio fundamental das definições circula em torno do conceito utilizado pelo Sebrae e da legislação específica do país, que é utilizada pela Receita Federal e está na Lei Geral para Micro e Pequenas Empresas. 

O número de empresas que divulga/comercializa produtos/serviços online é cada vez maior, mostrando que é uma oportunidade de negócio dinâmica e interessante, mas isso não quer dizer que seja simples manter um negócio online. O tempo de vida útil das empresas que comercializam seus produtos online é bem menor se comparado às empresas físicas. Desta forma, ganham cada vez mais relevância as discussões sobre as condições que levam a esse resultado, no sentido de garantir que não venham a acontecer com tanta frequência. 

O trabalho realizado ao longo deste estudo permitiu alargar os conhecimentos teóricos relacionadas aos fatores que exercem influência em uma empresa, sejam eles fatores específicos, externos e internos. Os fatores externos à empresa apresentam variáveis situacionais que podem facilitar ou inibir o empreendedorismo na inicialização e durante o ciclo de vida da PME. Fatores externos macroambientais não são controláveis e o sucesso da PME geralmente depende da capacidade do empresário de lidar com eles. O ambiente interno inclui todos os fatores específicos da empresa que são influenciados por ações específicas praticados por ela, incluindo a disponibilidade de recursos e habilidades pessoais para exercer funções empresariais e o uso efetivo de recursos dentro da empresa.

Os fatores habilitadores estão ligados às características específicas de cada negócio. Em contrapartida, existem autores que não estudam aspectos do negócio em si, mas sim traços de personalidade do dirigente do empreendimento ou seus hábitos e práticas que podem levar ao sucesso do negócio. Como por exemplo a orientação empreendedora do dirigente da empresa que será composta por algumas características que influenciam positivamente o sucesso do empreendimento, dentre elas, a adaptação à mudança, iniciativa, propensão a assumir riscos e capacidade de aprender.

A revisão bibliográfica identificou alguns fatores habilitadores, dentre eles, o conhecimento de tecnologia da informação por parte dos empresários/gestores; a terceirização de funções específicas, recursos ou infraestruturas de TI; o uso dos marketplaces; o gerenciamento do estoque com a utilização do método cross docking; atuar em um mercado de nicho, segmentado; utilizar instrumento indicador que monitora e acompanha o nível de desempenho (KPIs, satisfatório, ou insatisfatório) e o nível de atingimento dos objetivos organizacionais (KRIs) e por fim, realizar a análise ambiental e promover estudos de redesenho, para garantir assim a melhoria contínua dos processos da empresa. A atuação de cada influenciador vai depender do ramo de atividade da empresa, uma vez que cada negócio possui suas características específicas. 

O estudo mostrou que alguns fatores habilitadores identificados estão presentes em parte das empresas que participaram da pesquisa. Como por exemplo, as empresas com gestores que possuem  nível de conhecimento em tecnologia da informação “Bom” ou “Superior”, ou que utilizam algum instrumento indicador que monitora e acompanha o nível de desempenho (KPIs, satisfatório, ou insatisfatório) e o nível de atingimento dos objetivos organizacionais (KRIs), estão, em sua em sua maioria, em empresas que possuem maior tempo de mercado. As pessoas que informaram possuir nível de conhecimento em tecnologia da informação “Superior” utilizam algum instrumento indicador (KPIs, KRIs) em sua empresa e as organizações com maior tempo de mercado, em sua maioria, fazem o acompanhamento durante o período de pós-venda, monitorando o feedback dos clientes na internet.

Este trabalho contribui com a comunidade acadêmica ao estudar os fatores habilitadores candidatos para definição de um modelo de negócio eletrônico para empreendimentos de pequeno e médio porte que atuam no segmento de tecnologia da informação. Ou seja, buscou-se destacar a importância dos fatores relacionados ao sucesso, ainda que de forma parcial. Além disso, este trabalho contribui ao meio acadêmico pela abordagem utilizada, uma vez que a pesquisa buscou obter a visão dos proprietários em detrimento da experiência dos usuários, essa mais frequentemente utilizada na avaliação da atuação das empresas.


\section{Limitações}
\label{sec:model}

Uma das limitações da pesquisa utilizada é inerente ao uso dos questionários como estratégia de pesquisa. Trata-se da baixa representatividade de empresas estudadas frente ao todo. Ou seja, o estudo de 55 casos, e casos estes somente brasileiros, apresenta-se como uma deficiência do trabalho.

\section{Trabalhos futuros}
\label{sec:model}

Na sequência do presente trabalho surgiram alguns aspectos que se revelaram interessantes para uma abordagem mais detalhada. De seguida, são referidos sumáriamente aqueles que poderão vir a ser objeto de futura investigação:
\begin{itemize}
\item Para trabalhos futuros seria interessante identificar como surgem e como se formam novos modelos de negócios nos diferentes setores da economia. 
\item Poderá desenvolver este trabalho numa perspetiva mais abrangente, selecionando para tal, uma amostra estatisticamente significativa, que poderia abranger PMEs que atuam de forma eletrônica em uma região geográfica maior.
\end{itemize}
