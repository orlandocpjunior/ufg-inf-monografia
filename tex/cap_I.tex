
\chapter{Introdução}
\label{cap:intro}

O mercado brasileiro de e-commerce está em alta, mostrando que é uma oportunidade de negócio dinâmica e interessante, mas isso não quer dizer que seja simples manter um negócio online. Pesquisas mostram que as páginas dedicadas ao comércio eletrônico no país não costumam funcionar por muito tempo \cite{ecomnews} \cite{ecommerceschool2015}. Um levantamento feito pela empresa BigData Corp apurou que no ano de 2017 o tempo de vida dessas lojas foi em média, de 185 dias. Esse resultado foi obtido a partir da captura de dados em mais de 20 milhões de endereços brasileiros da internet \cite{sbvcsociedadebrasileiradevarejoeconsumo2017}.

A incapacidade de obter lucros é frequentemente vivenciada pelas organizações que trabalham com a internet no Brasil. No ano de 2013 por exemplo, a Dafiti, maior loja virtual de moda do Brasil, apresentava um prejuízo que representava cerca de 50\% o valor de seu faturamento. O número de varejistas on-line que crescem no Brasil, porém não lucram, é grande e traz nomes de organizações reconhecidas no mercado como Mobly, Casas Bahia, Ponto Frio, Netshoes \cite{rosanasantarosa2016}. 

Assim, podemos perceber que para muitas empresas a internet ainda não cumpriu suas promessas. Embora fazer negócio no ambiente digital possa ser original e estimulante, também pode ser frustrante, confuso e não lucrativo \cite{albertina.l.2010}. A B2W, grupo controlador das marcas Submarino, Americanas e Shoptime possue resultado que só vem decrescendo com o passar do tempo, deixando a mesma de operar com lucro em 2009 para adentrar em grandes prejuízos nos anos subsequentes, sendo que em 2011 o total de perdas alcançou o patamar negativo de R\$ 30 milhões, em 2012, de 42,83 milhões, e em 2013, de 10 milhões \cite{freitasc.2014}. Nestas grandes organizações o prejuizo pode estar de acordo com os planos traçados pelos gestores, com o objetivo de conquistar um lugar no mercado, o que não vem a ser o caso das PMEs.

Um estudo realizado sobre sobrevivência e mortalidade das empresas constatou que mercados que apresentam elevada taxa de entrantes são também aqueles com as maiores taxas de saída \cite{mourao2010}. Por volta do ano 2000 muitas empresas que atuavam na internet tinham modelos de negócios inviáveis baseados em expectativas oriundas do que era considerada a nova tecnologia \cite{motta2013}. Entre 2000-2001, em meio à explosão da bolha da internet, mais de 600 empresas on-line fecharam nos Estados Unidos e mais de 1.000 ao redor do mundo \cite{rosanasantarosa2016}. Essa quantidade de empresas pode parecer pequena, mais representa uma porcentagem significativa se comparada a quantidade de empresas que atuavam neste setor naquele momento. 

Apesar da crescente participação das empresas no comércio eletrónico, a grande maioria delas descobre que a simples presença na internet não se traduz em sucesso automático \cite{rosanasantarosa2016}. A pesquisa Profissionais de Ecommerce realizada pela Ecommerce School revelou que de uma base de aproximadamente 23 mil lojas virtuais no ar no Brasil, 70\% delas estavam abandonadas, ou seja, embora on-line não investiam em marketing para atrair visitantes e consequentemente não vendiam \cite{ecommerceschool2015}. 

Se um site não recebe visitas, os resultados são rapidamente sentidos pelos administradores do empreendimento virtual, podendo levar a uma decisão de renunciar aos esforços da empresa na internet \cite{rosanasantarosa2016}. Desta forma, ainda que novas formas de se conduzir os negócios tenham surgido, os fundamentos da competitividade empresarial se mantêm inalterados, seja para pequenos ou grandes empreendimentos \cite{porter2001}. 

As Pequenas e Médias Empresas PME’s enfrentam grandes obstáculos nesse ambiente, principalmente porque são caracterizadas por terem baixa intensidade de capital, poder decisório centralizado e a forte presença dos proprietários e familiares como mão-de-obra \cite{IBGE2003}. Esses empreendimentos não possuem uma definição específica. Existe uma variedade de critérios para essa classificação, ora pela legislação específica que leva em conta o faturamento da empresa, ora pelo Sebrae que leva em conta o número de pessoas ocupadas.

Alguns estudos relatam que fatores habilitadores para definição de um modelo de negócio eletrônico como estratégias a serem empregadas para o sucesso \cite{criticalfactors2012}. Esses fatores englobam um grupo restrito de áreas da empresa, de práticas ou de cargos ocupados considerados como chave para o sucesso de uma organização \cite{criticalfactors2014}. Desta forma, a definição da lógica de captura, entrega e proposição de valor de um negócio eletrônico pode impactar no índice de mortalidade dos negócios eletrônicos?

\section{Objetivos}
\label{subsec:framing}
%% - - - - - - - - - - - - - - - - - - - - - - - - - - - - - - - - - - --

Nesse contexto, o objetivo deste estudo é identificar quais são os fatores habilitadores candidatos para definição de um modelo de negócio eletrônico para empreendimentos de pequeno e médio porte que atuam no segmento de tecnologia da informação.

\section{Objetivos Específicos}
\label{subsec:framing}
%% - - - - - - - - - - - - - - - - - - - - - - - - - - - - - - - - - - -

Para alcançar o objetivo geral, os seguintes objetivos específicos foram definidos:

\begin{enumerate}
\item Identificar as causas que afetam o tempo de vida útil das PME´s no cenário eletrônico.
\item Descrever as características dos negócios digitais e os motivos que conduzem à sua ascensão.
\item Apresentar os princípios e abordagens do Modelo de Negócio - Canvas. 
\item Discorrer sobre os fatores internos e externos que exercem influência nas organizações.
\item Analisar e discutir os elementos contextuais que conduzem aos fatores habilitadores para definição de um modelo de negócio eletrônico.
\end{enumerate}

A estrutura e a organização deste trabalho foram delineadas da seguinte forma: o Capítulo \ref{cap:theory} apresenta a revisão da literatura, descrevendo o modelo de negócio, as características das pequenas e médias empresas e os fatores internos e externos que podem influenciar no seu sucesso; o Capítulo \ref{cap:texto} descreve a metodologia, mostrando como foi a trajetória da pesquisa até chegar ao resultado final; o Capítulo \ref{cap:results} apresenta os resultados obtidos  e o Capítulo \ref{cap:conclusion} apresenta a conclusão sobre este trabalho.





