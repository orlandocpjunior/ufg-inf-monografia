\chapter{Metodologia}
\label{cap:texto}

Trata-se de um estudo exploratório, feito através de uma revisão bibliográfica com fontes secundárias e posteriormente com pesquisa quantitativa. A população alvo deste projeto de estudo são empresas de pequeno e médio porte que atuam de forma eletrônica. As buscas foram realizadas em sua maioria nas seguintes bases de dados bibliográficas: Google Scholar, SciELO, Periódicos Capes, ArXiv e Microsoft Academic. Ao finalizar as pesquisas em cada base, as referências duplicadas foram excluídas. 

Foram selecionados artigos publicados entre 2000 e 2019 (incluindo aqueles disponíveis online em 2019 que poderiam ser publicados em 2020), escritos em inglês, português ou espanhol. Optou-se pela busca por termos livres, com essa estratégia, houve uma recuperação de um número maior de referências, garantindo a detecção da maioria dos trabalhos publicados dentro dos critérios pré-estabelecidos. Os termos e-business; modelo de negócio; SME digital business; e critical success factors foram combinados com as associações e desfechos de interesse.

\section{Coleta de dados para a pesquisa}
\label{sec:figs}
 
Para obter uma amostra foi definida uma população alvo. A população alvo é o grupo ou os indivíduos a quem a pesquisa se aplica. Idealmente, a população alvo foi representada por uma lista de 285 empresas. A ferramenta para coleta de dados é um questionário estruturado que busca identificar o quanto estas empresas estão aderentes aos fatores habilitadores identificados.

Os questionários foram distribuídos via e-mail para empresas que de alguma forma atuam em parceria com o Instituto de Informática da Universidade Federal de Goiás. Foram aplicados também questionários em um workshop promovido pelo Instituto de Informática, em que foi apresentado a comunidade empresarial a Lei Geral de Proteção de Dados. As empresas também foram contatadas de forma aleatória por conveniência, validando a participação daquelas que concordaram em fazer parte do estudo colaborativamente.

A pesquisa de opinião aplicada ficou acessível na ferramenta Google forms por 29 dias, de 10/10/2019 a 07/11/2019, através do link: https://forms.gle/G5Pafno4JrR7NCpz7. Da relação total de convidados via e-mail, 41 responderam ao formulário, resultando em índice de retorno dos questionários de 15,41\%, o que pode ser considerado um número baixo, dado que para Marconi e Lakatos, questionários que são enviados para os entrevistados alcançam em média 25\% de devolução \cite{marconilakatos2005}. Dentre as principais desvantagens das pesquisas on-line podemos considerar a baixa taxa de resposta aos questionários \cite{henrique2010}.

Ao todo obteve-se 58 questionários respondidos, seja através dos questionários on-line e aplicação dos questionários presenciais. Algumas respostas foram dadas por indivíduos pertencentes a mesma organização, por isso houve a ocorrência de mais de um respondente para uma mesma empresa, reduzindo assim para um total de 55 questionários de pesquisa considerados válidos.


\section{Análise estatística}
\label{sec:figs}

O tratamento dos dados se dará através da ferramenta Excel, Google Sheets e do classificador do programa Weka 3, em que as respostas serão organizadas e, em seguida, através de tabelas e gráficos serão exibidas as informações provenientes das respostas coletadas. 

Do total de questionários considerados válidos, 15 responderam que não comercializam produtos/serviços em canais eletrônicos e 40 disseram que a empresa comercializa seus produtos/serviços em canais eletrônicos (e-commerce). Desta forma, com base nos resultados obtidos destas 40 empresas, foram feitos alguns gráficos para representar as informações, como mostrado no próximo capítulo.





